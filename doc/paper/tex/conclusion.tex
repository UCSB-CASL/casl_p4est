%!TEX root = draft.tex
\section{Conclusions}
In this article we have presented parallel algorithms related to the level-set technology on adaptive Quadtree and Octree grids using a domain decomposition approach. These algorithms are implemented using a combination of \texttt{MPI} and the open-source \texttt{p4est} library. In order to preserve the unconditional stability property of the semi-Lagrangian scheme while enabling scalable computations, we introduced an asynchronous interpolation algorithm using non-blocking point-to-point communications, and demonstrated its scalability. 

In particular we showed that the scalability of the semi-Lagrangian algorithm, depends on the CFL number. Great scalability is observed for intermediate CFL numbers, e.g. $\text{CFL}\sim10$. At higher CFL numbers, however, the departure points are potentially further dispersed across processors, which limits the scalability. This is because the domain decomposition technique used here is based on the Z-ordering of cells and does not take the velocity field information into account. A possible remedy for this problem could be assigning weights to cells based on some estimate of the grid structure after one step of the advection algorithm, e.g.\ by using a forward-in-time integration of grid points. Such an estimate could also reduce the number of semi-Lagrangian iterations. These ideas are postponed for further investigations. We have also presented a simple parallelization technique for the reinitialization algorithm based on the pseudo-time transient formulation. Both the semi-Lagrangian and the reinitialization algorithms show good scalability up to 4096 processors, the current limit of our account. 

Finally, an application of these algorithms is presented in modeling the solidification process by solving a Stefan problem. This application clearly illustrates the applicability of our algorithms to complex multi-scale problems that cannot be treated practically using the domain decomposition techniques on uniform grids. We believe that our findings can serve as a basis to simulating a wide range of multi-scale and free boundary problems.

\section*{Acknowledgment} 
The funding for this project was provided by ONR N00014-11-1-0027.
We would also like to thank the technical staff at the Texas Advanced Computing
Center (TACC) and the developer community of PETSc library for their valuable
suggestions during this project.
The third author acknowledges support by the Hausdorff Center for Mathematics
(HCM) at Bonn University,
% and the Transregio 32 research collaborative, both
an initiative funded by the German Research Foundation (DFG).
