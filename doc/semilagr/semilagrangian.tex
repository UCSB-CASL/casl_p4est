
\documentclass[11pt]{article}

\usepackage{algorithm,algorithmic}
\usepackage{amsmath}
\usepackage[margin=1in]{geometry}
\usepackage{xspace}

\newcommand{\pforest}{\texttt{p4est}\xspace}

\setlength{\emergencystretch}{20pt}

\title{A scalable semi-Lagrangian method on octree meshes}
\author{}

\begin{document}

\maketitle

\begin{abstract}
We present a semi-Lagrangian method for solving the advection equation
on nested non-conforming hexahedral meshes.
\end{abstract}

\section{Introduction}

\section{The semi-Lagrangian method}

\begin{algorithm}
  \caption{Basic semi-Lagrangian time step}
  \begin{algorithmic}
    \FOR{$e$ in local cells}
      \FOR{$n$ in nodes on $e$ that have not yet been processed}
        \STATE Extrapolate location of $n$ backward in time as $x_n$
	\IF{$x_n$ is located on a different processor $p_n$}
	\STATE Append $x_n$ to send buffer for $p_n$
	\ENDIF
      \ENDFOR
    \ENDFOR
    \STATE Reverse communication pattern so the receivers know the senders
    \STATE Send the non-local points $x_n$ to the target processors
    \STATE Loop over local points $x_n$ and interpolate new values at $n$
    \STATE Receive points $x_n$, interpolate, add to send buffer
    \STATE Send back the interpolated values
    \STATE Receive values for non-local points and update node values
  \end{algorithmic}
\end{algorithm}

\end{document}
