\documentclass{scrlttr2}
\usepackage{color}

\begin{document} 
\begin{letter} {Editorial Office, \\ Journal of Computational
Physics} 
\setkomavar{subject}{Re: JCOMP-D-15-00846} 

\opening{Dear Editor,} 
We would like to appreciate the invaluable comments made by the reviewers. We
have made adjustments to our article, titled `\textbf{Parallel level-set methods
on adaptive tree-based grids}', to address the issues raised by the reviewers as
detailed below. 

\begin{itemize} 
\item \textbf{Reviewer \#1} \\ 
The first reviewer has suggested to include ``\textit{some basic library codes
in appendix, each matching its level set algorithm}''. Indeed it is important
for the scientific community to share knowledge and facilitate ways for
reproducing scientific work for others. In the case of computational science,
unfortunately, this is a daunting task that is not always possible.  Many
scientific codes are highly specialized and are often too technical to be fitted
into a paper. 

Specifically, in our case, our library amounts to nearly 30,000 lines of code,
spread across more than 50 files and with complex dependencies. As such, it is
practically not viable to include sample codes in the paper. Moreover, we
believe inclusion of actual code, which involves many technical details, e.g.
error checking, profiling tools, comments, etc., is not scientifically
educational and may further alienate the reader from the main message of the paper.

Instead, and in accordance to the second reviewer who has raised related issues,
we have opted to revise the pseudo-codes in the algorithms to make them more
clear. We believe that the modifications should make the algorithms easier to
understand and more useful to the readers.

\item \textbf{Reviewer \#2} \\ 
The second reviewer has raised several issues, mostly related to the
pseudo-codes in the paper, as well as suggesting additional numerical tests.
These issues are addressed as detailed below: 

\begin{itemize} 
\item ``\textit{It took me a while to understand Algorithms 1 and 2. It is recommended that the authors�� pseudocode be closer to a language that many
people are familiar with like C++ or Fortran 90 (or perhaps, the authors’ are
using a new language that this referee is not aware of?).}''

Our library is coded in `\texttt{C++}' (with a mix of little `\texttt{C}' code).
We have made adjustments to the to algorithms to make them easier to understand.
However, we believe it is best not to directly use a `\texttt{C++}' syntax to
keep the pseudo-code language-agnostic and accessible to all readers.

\vspace{.5cm}

\item ``\textit{In Algorithm 2, what is the definition of “rank”? “owner’s
  rank”? “mpirank”? “st”?}''

There were multiple references to ``\textit{rank}'' in the pseudo-code that 
would have caused confusion. We have made things clear by explicitly referring 
to ``\textit{owners\_rank}'' and ``\textit{mpirank}''. Also in the previous
pseudo-code ``\textit{st}'' was meant to indicate ``\texttt{MPI\_STATUS}''. Again
we have reorganized the pseudo-code to make this point clearer. We have also
added more comments and text to the caption to explain things in more details.

\vspace{.5cm}

\item ``\textit{In algorithm 2, if X is contained within the mesh on a given
  processor, does that mean all the points needed in order to derive the value
at X are also on the same processor? What if the points in the interpolatory
stencil of X belong to multiple processors? Which processor does the actual
interpolation? Is it the processor that needs the value at X? or is it the
processor that owns X?}''

Since the $Z$-curve defines non-overlapping partition of the grid in parallel, it
is guaranteed that the rank found by the search operation in line 3 of algorithm
2, actually owns the interpolation point, i.e. there exists a \textit{local
cell} $\mathcal{C}$ belonging to process \texttt{owners\_rank} which contains the
interpolation point. This is because the search operation uses a binary search
on the $Z$-curve to find the remote \texttt{owners\_rank}.

Our method for interpolation only requires that the values of the function (and
optionally its second derivatives if using multi-quadratic interpolation) be
defined on all vertices of cell $\mathcal{C}$. This, in turn is guaranteed by the
fact that the boundary of each process is covered by a layer of ghost cells that
have updated the ghost values at all their vertices before performing the
interpolation. As a result, all the vertices of cell $\mathcal{C}$, whether local
or ghost, have valid values that can be used for interpolation.

\vspace{.5cm}

\item ``\textit{In section 4.2, the authors should do a standard 3D reversible
  test when checking speed-up. A standard test allows researchers to compare
the new method to previous schemes. Also the authors should report the
numerical error (e.g. the symmetric difference error or the level set error in
proximity to the zero level set). For a reversible problem, the exact solution
is known at the end time. So in Table 2, an extra column should be added to
both the (a) and (b) parts which has the error (it is assumed that the error is
independent of p?). For the reversible problem, the zero level set isosurface
should be displayed at maximum deformation t = T/2 and at the end time t = T
for some representative values of “CFL” and lmax.}''

The idea behind the test in section 4.2 was to look at the individual
components that make the semi-Lagrangian algorithm and check their scaling and
see how different parameters, most notably the CFL number, would affect the
scaling results. In doing so we decided to look at the performance of a single
iteration of semi-Lagrangian algorithm in isolation without including other
ingredients, such as reinitialization, that would otherwise affect the timing
and scaling. In fact the scaling results for a more comprehensive test are
included toward the end of article and in the Stefan test.

Also, our main focus in this article has been the scalability of several level
set algorithm in parallel. The accuracy of many of these algorithms, albeit in
serial, have been studied before in relevant articles that we have cited in
details. However, we agree with the reviewer that it is important to show that
the presented methods are in fact accurate and convergent. As a result we have
added a new section, 3.4, that checks the accuracy of the standard rotation
test as was suggested by the reviewer.

\vspace{.5cm}

\item ``\textit{Also in section 4.2, how frequently is reinitialization carried
  out? What is the pseudo time step $\Delta \tau$? What is $\tau_{stop}$? If
there is no reinitialization for this test, it is requested that the
semi-Lagrangian algorithm together with reinitialization be analyzed in this
section too}''

Previous studies have shown that the adaptive time-stepping method that is used
in this article greatly improves the convergence to steady state solution of
the reinitialization equation (c.f. ref 33 in main text). We have generally
observed that a few, typically 20, iterations of the algorithm is enough for
the solution to converge within a small band (e.g. one smallest cell in the
diagonal direction, i.e. $|\phi| < \sqrt{2}\Delta x_{min}$) around the
interface. As a result we solve the reinitialization equation not to a final
time (since each cell is using a different time-step) but to a fixed number of
iterations. We have added a new paragraph in section 3.3 that clarifies this
approach.

As for the test in section 4.2, there is no reinitialization involved. This is
because we are testing the effects of various parameters on the scalability of
a single iteration of the semi-Lagrangian. We believe, however, that the
reviewer is absolutely correct about the need to have a test of the entire
algorithm. However, we argue that this role is already fulfilled by the Stefan
test toward the end of the article. 

The Stefan test of section 5 is a complete and non-trivial example and
represents a typical scenario that one should expect when using our algorithms
in practice. This example serves, not only as a demonstration of what our
algorithms can achieve, but also as a scalability test of all different
components, such as interpolation, semi-Lagrangian advection, reinitialization,
solution of implicit linear system, and extension of this solution over the
interface. It also compares these costs with costs associated with grid
operations in its entirety which includes generation, refinement and
coarsening, and partitioning of the forest. As figure 10, and table 7,
illustrate, both the semi-Lagrangian and the reinitialization scale well in a
real scenario. Thus we hope the results of this test is satisfactory enough for
the reviewer.
\end{itemize} 
\end{itemize}

In light of these adjustments, we respectfully ask that our article be reconsidered for publication in the Journal of Computational Physics.
\closing{Sincerely,} 
Mohammad Mirzadeh (on behalf of all authors) 
\end{letter}
\end{document}
